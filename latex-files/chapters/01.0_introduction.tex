Over the years, the architecture of many large applications has evolved from monoliths to microservices in order to make the IT environment more flexible and responsive to today's fast-changing business requirements. The microservices architecture, inspired by the principles of Service-Oriented Architectures (SOA) \cite{dragoni2017yesterday}, consists of small autonomous applications that communicate together with lightweight mechanisms like HTTP requests. A microservice is a standalone application that can be independently deployed, scaled, and tested and has a single responsibility \cite{thones2015microservices}. This means that a microservice should be fully controlled and owned by a single team. To better understand the fundamentals of microservices, we first discuss the notion of monolithic systems. \par
In a monolith, all code is combined into a single executable. A monolith is relatively easy to build, but when the application gets bigger, it becomes difficult to understand and maintain the code. \citeauthor{fritzsch2019microservices} \cite{fritzsch2019microservices} researched the intentions, strategies, and challenges of microservices and found out that the main driver to migrate from a monolith to microservices is the lack of code maintainability. The lack of maintainability often occurs when the codebase becomes too big, making it difficult for developers to keep track of changes. \citeauthor{kalske2017challenges} \cite{kalske2017challenges} define three drivers for migrating to microservices, namely: a high number of teams, a big codebase, and teams geographically located far from each other. When none of the aforementioned drivers exists, microservices might not be the most appropriate solution. According to \cite{kalske2017challenges}, it is preferred to start with a monolith in order to avoid technical and organisational challenges that microservices bring. For this reason, most companies start with a monolith and iteratively migrate towards a microservices architecture when the codebase becomes unmanageable \cite{fritzsch2019microservices, kalske2017challenges}. \par 