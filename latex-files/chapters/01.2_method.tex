\section{Research method}\label{s:method}
The research method that we use throughout this research relies on characteristics of Design Science (DS). Design science is one of the main research paradigms in the Information Systems (IS) discipline \cite{hevner2004design}. The paradigm focuses on the creation and evaluation of IT artifacts that aim to solve identified organisational problems \cite{hevner2004design}. There exist many IS design science frameworks in the current literature, and they all have small adaptations. \citeauthor{offermann2009outline} \cite{offermann2009outline} conducted a comparison of design science methodologies and discovered three main phases: (1) problem identification, (2) solution design, and (3) evaluation. \par

Throughout this research, we follow the Design Science Research Methodology (DSRM) introduced by \citeauthor{peffers2007design} \cite{peffers2007design}. There are three reasons for choosing this methodology. First, the methodology has an objective and scientific nature \cite{venable2017choosing}. This means, e.g., that the problem statement is mostly based on literature and not derived from subjective opinions from stakeholders. Moreover, the DSRM method is made for IS research that aims to design artefacts that are tools or techniques rather than products or processes \cite{venable2017choosing}. In this research, we aim to design such a technique. At last, the methodology does not consider the designed artefact to be implemented in the real world \cite{venable2017choosing}, which is in line with our research. \par

The DSRM is synthesised from seven other design science methods and consists of six activities executed in a nominal sequence. Table \ref{tab:research_approach} gives an overview of the steps and the related research methods and questions. The activities are described below. \par

\begin{table}[h]
    \small
    \caption[Overview of research process]{Overview of the research process and the related research questions and methods that are used.}\label{tab:research_approach}
    \begin{tabular}{>{\raggedright}m{75pt} >{\raggedright}m{107pt} >{\raggedright}m{60pt} >{\raggedright\arraybackslash}m{100pt}}
        \toprule
        Phase by Offerman et al. 
        & Activities by Peffers et al. 
        & Question(s) 
        & Research Method \\
        \midrule
        Problem Identification 
        & Problem identification and motivation 
        & SQ1
        & Literature Study I \\
        & Define the objectives for a solution 
        & SQ2, SQ3 & Literature Study II \\
        \midrule
        Solution Design 
        & Design and development
        & -
        & Prototyping \\
        \midrule
        Evaluation 
        & Demonstration 
        & SQ4, SQ5
        & Experiment I \\
        & Evaluation & MRQ & Experiment I \\
        \midrule
        - & Communication & MRQ & - \\
        \bottomrule
    \end{tabular}
\end{table}

% h = float specifier
% 312pt is the full-width of the table

\begin{itemize}
    \item \textbf{Problem identification and motivation.} During the problem investigation phase, the goal is to \textit{specify the research problem and justify the value of a solution} \cite{peffers2007design}. This is done by conducting a systematic literature review. This first literature review is needed to collect and review the state-of-the-art concerning the decomposition techniques. The literature review, presented in Chapter \ref{c:related_work}, results in a problem statement (see Section \ref{s:problem}) in which the importance of the solution is clarified.
    \item \textbf{Define the objectives for a solution.} In the next phase, the objectives of the designed artefact are defined. To do this, we perform a second literature review in which potential solutions are studied in a semi-systematic way. For instance, we look thoroughly into clustering algorithms that are particularly used in this domain in order to collect requirements. Furthermore, we define the context in which the solution is executed. This means, e.g., we examine tools and libraries to extract data from the monolith and define which applications are suitable to validate the technique. We also use the results of the first literature review to get insights into the state-of-the-art. 
    \item \textbf{Design and development.} In this phase, we design and develop a prototype of the multi-view decomposition technique. The prototype is verified on toy examples in order to prove it works as intended. Chapter \ref{c:multi-view_clustering} describes how the multi-view clustering tool is developed, and the design decisions that are made during the process.
    \item \textbf{Demonstration.} In the demonstration step, the multi-view clustering tool is validated in a more realistic context. The applications found during the second literature review are used to demonstrate the working of the prototype on another \textit{instance of the problem} \cite{peffers2007design}. To do this, we apply the tool on seven open-source projects. For each project, the tool extracts static, semantic, and dynamic dependencies used to decompose the system. As input for the decomposition, we use three different levels of information: a single data source, a pair of data sources, and all the data sources. This leads to a total of seven distinct decompositions. An overview of the experiment is presented in Table \ref{tab:experimental_design}. The stars ($*$) in the table indicate which source of information is active in each group.
    \begin{table}[h]
\centering
    \caption[Experimental design]{Experimental design. Each star (*) in the table indicates the appearance of the data source while an empty cell shows the absence of the data.}\label{tab:experimental_design}
    \begin{tabular}{c|c|c|c|c|c|c|c|}
        \cline{2-8}
        & \multicolumn{7}{c|}{Experiments}\\
        \cline{2-8}
        & 1 & 2 & 3 & 4 & 5 & 6 & 7 \\
        \hline
        \multicolumn{1}{|c|}{Static} & * & & & * & & * & *\\
        \hline
        \multicolumn{1}{|c|}{Dynamic} & & * & & * & * & & *\\
        \hline
        \multicolumn{1}{|c|}{Semantic} & & & * & & * & * & *\\
        \hline
        & \multicolumn{3}{c|}{Single} & \multicolumn{3}{c|}{Pair} &  \multicolumn{1}{c|}{All}\\
        \cline{2-8}
    \end{tabular}
\end{table}
    %\begin{table}[h]
    \small
    \caption[Experimental design]{Experimental design. Each star (*) in the table indicates the appearance of the data source while an empty cell shows the absence of the data.}\label{tab:experimental_design}
    \begin{tabular}{r|c|c|c|c|c|c|c}
        \toprule
        \multirow{3}{*}{Information used}& \multicolumn{7}{c}{Experiments}\\
        \cmidrule{2-8}
        & \multicolumn{3}{c|}{Single} & \multicolumn{3}{c|}{Pair} &  \multicolumn{1}{c}{All}\\
        \cmidrule{2-8}
        & 1 & 2 & 3 & 4 & 5 & 6 & 7 \\
        \midrule
        Static dependencies & * & & & * & & * & *\\
        \midrule
        Semantic similarity & & * & & * & * & & *\\
        \midrule
        Dynamic dependencies & & & * & & * & * & *\\
        \bottomrule
    \end{tabular}
\end{table}
    \item \textbf{Evaluation.} In the evaluation step, we analyse the different decompositions collected from the experiment in order to quantify the quality of the decomposition. The different decompositions are compared to each other in order to gain an understanding of the effect of information sources on the quality of the decomposition. The evaluation is done by microservice specific measures that are researched and selected during the second literature review. 
    \item \textbf{Communication.} In the last phase, we present the research results to interested fellow students and, depending on the quality, made it ready for a scientific publication.
\end{itemize}
